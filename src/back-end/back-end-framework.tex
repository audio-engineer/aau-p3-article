% textidote: ignore begin
\subsection{Back-end framework}\label{subsec:back-end-framework}
% textidote: ignore end

The framework used in the back end is Spring Boot~\cite{springboot2024}.
Spring Boot is an open-source framework for the Java programming language and is built on top of the Spring framework.
The reason for choosing this approach is rather obvious as this framework alone encapsulates all features needed for our
application out of the box.
It provides a relatively easy way to develop a REST API for web applications, as it has built-in support for the MVC
architecture.
In relation to the MVC architecture, we use Spring Boot for the controller and model.
With Spring Boot, we use the Gradle Build Tool~\cite{gradle2024} for Java to handle dependency management, incremental
builds, build caching among other important things.
