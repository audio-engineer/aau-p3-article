% textidote: ignore begin
\section{Process Analysis}\label{sec:process-analysis}
% textidote: ignore end

The process analysis is an important tool in improving the process of future projects and team work.
This process analysis covers the process of developing the P3 program as well as the report.

The authors consist of five students.
These students also worked together on the P2~\cite{P2}, P1 and P0 projects.
Because of this, the students are familiar with each other and thereby are aware of the strength and weaknesses of
members within the group.

Both work load balancing and work load distribution were not optimal.
It can be seen in the code frequency that most code, both for the report and the software, is written in periods of
concentrated effort.
This could be because people are more motivated to work when others are.
Optimally, the code should be written throughout the semester somewhat evenly to make sure the work load was more
balanced.
Work distribution is also a factor, as not all members contribute the same amount of work both for the report and the
software.
The reasons for this could be that some people are more motivated to work on the project or that some people have an
easier time writing.
Once again, it would be optimal if all members contributed the same amount.

Even though most of the team work went well, there were some challenges in the beginning.
At the beginning of the semester, the group consisted of six members.
As the group uses several technologies to aid their work, as also described in Section~\ref{subsec:development}, the
group had to teach the new sixth member how to use these and how they worked.
This hindered the efficiency in the beginning, but was meant to aid it further on in the semester.
The sixth member, however, eventually began to not show up nor respond to messages, which is why the group had to remove
the person.
Removing the team member was necessary but also took up time in supervisor meetings and intergroup meetings throughout
the semester, ultimately hindering productivity.

Earlier projects have been somewhat out of scope when it comes to complex integrations.
However, the group has learned from this and down scaled the complexity for P3 to only contain the suggested elements.
The less complexity allows the group to get a deeper understanding of the suggested technologies.
This is both good and bad, as it is also interesting to learn other technologies whilst working on the semester
projects.

The main issue when working together has been planning.
In P3, planning has gone significantly better.
The code was successfully implemented a few weeks in advance of the hand-in deadline allowing the group to test it
thoroughly in this period.
This time advantage also meant that the group was able to focus on the report in the weeks leading up to the hand-in
deadline and make sure it was written as expected.

In the future, the group should consider using workload tracking tools or strategies for more even contributions.
This could be done through better scheduling of work and incorporating milestones into the workflow.
