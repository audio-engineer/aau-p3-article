\subsection{Important back-end functions}\label{subsec:important-back-end-functions}

Spring Boot supports features such as repositories, services, batch-processing, and more.
These are the pillars on which our application stands why it is important to outline them.
Described in the following subsections are examples of these features.

\subsubsection{Controller}

A rest controller in Spring is a class denoted by having the~\texttt{@RestController} annotation.
A controller is the first point of contact when engaging with the REST API\@.
In the NOVA Dashboard application, there are two controllers.
One of them with two post-mappings for the orders-file and order-lines-file respectively, as seen in
Listing~\ref{lst:controller}.
This controller handles all HTTP POST requests to the URLs~\url{/api/orders} and~\url{/api/order-lines}.
It expects files and returns a~\texttt{ResponseEntity} object on success.
In the listing is only included the route for the order-files as they both are handled almost identically by the
controller.

\begin{lstlisting}[
    style=java,
    caption=A selection of code from the import controller in the application.
    Dots (...) represent skipped code.,
    label={lst:controller}]
@RestController
@RequestMapping("/api")
...
public class ImportController {
    ...
    @PostMapping(value = "/orders", consumes = MediaType.MULTIPART_FORM_DATA_VALUE)
    public final ResponseEntity<Map<String, String>> orderHandler(
        @Valid @RequestParam("orders") final MultipartFile multipartFile) {
            ...
    }
}
\end{lstlisting}

\subsubsection{Repository}

The repository manages access to the data layer.
It is an invaluable abstraction that makes the interaction with the database seamless.
The repository is then further abstracted by extending the JPA repository library.
The JPA repository has built-in common functionality such as all CRUD related methods and so on.
This means that we only need to write uncommon queries.
Furthermore, JPA is able to recognize what functionality such a query will entail,
why it is also unnecessary to code the functionality ourselves.
One simply has to define the functionality in the name of the function as seen in Listing~\ref{lst:repository}.

\begin{lstlisting}[
    style=java,
    caption=A selection of code from the order repository in the application.,
    label={lst:repository}]
public interface OrderRepository extends JpaRepository<Order, Long> {
   Optional<Order> findByOrderId(UUID orderId);
}
\end{lstlisting}

\subsubsection{Model}

The model describes the core data structure of the application.
In Listing~\ref{lst:model} is included some fields of the model for an order.
Field examples included are:

\begin{itemize}
    \item \texttt{`created'} which holds information about the time of creation of the object in the application.

    \item \texttt{`price'} which is derived from the data uploaded by the user.

    \item \texttt{`orderId'} which is the unique identifier for the specific object.

    \item \texttt{`orderLines'} which is the one-to-many mapping rule between the order and order-line models.
\end{itemize}

Furthermore, the annotations seen in the listing abstract away common functionality within a model.
Examples of these are the~\texttt{@Getter} and~\texttt{@Setter} annotations.
These make it such that it is not necessary to explicitly write getters and setters for each field.

\begin{lstlisting}[
    style=java,
    caption=A selection of code from the order model in the application.
    Dots (...) represent skipped code.,
    label={lst:model}]
@Entity
@Table(name = "order")
@NoArgsConstructor
@AllArgsConstructor
@Getter
@Setter
@ToString
public class Order {
    @Column(nullable = false)
    @NonNull
    private ZonedDateTime created;
    ...
    @Column(nullable = false)
    @NonNull
    private BigDecimal price;
    ...
    @Id
    @Column(nullable = false)
    @NonNull
    private UUID orderId;
    ...
    @OneToMany(mappedBy = "order", cascade = CascadeType.ALL, orphanRemoval = true)
    private List<OrderLine> orderLines;
}
\end{lstlisting}

\subsubsection{Batch processing}

The batch processing is an integral part of the back end, as it is the most complex.
In the application, it is mostly used as a tool to validate and save data in parallel.
As the process is rather complicated, it is abstracted into different functions and classes.
The process is instantiated in the controller.
Hereafter the `ImportJobService' service handles the execution.
The service derives functionality from several configuration classes.
An example of a such configuration class is seen in Listing~\ref{lst:batch-processing}.
Herein it can be seen that a Spring Boot Job for orders among others is configured.
Its main purpose is that it builds a Spring Boot JobBuilder.
It provides a wrapper step called `orderImportStep' that contains other steps.

\begin{lstlisting}[
    style=java,
    caption=A selection of code from the batch processing logic in the application.
    Dots (...) represent skipped code.,
    label={lst:batch-processing}]
@Configuration
@NoArgsConstructor
@ToString
class JobConfiguration {
    @Bean
    @OrderImportJob
    Job orderImportJob(
        final JobRepository jobRepository, @Qualifier("orderImportStep") final Step orderImportStep) {
    return new JobBuilder("orderImportJob", jobRepository).start(orderImportStep).build();
    }
    ...
}
\end{lstlisting}

\subsubsection{Service}

Lastly, the service layer will be described.
The service layer encapsulates business logic and is the component that coordinates between the other components such as
controllers, repositories, and models.
An example of this is the `ImportJobService' as seen in Listing~\ref{lst:service}.
This service contains four different functions.
It incorporates the repository at the start, is called by the `ImportController' and uses the functionality from the
batch processing configurations.
Specifically, the core functionality in this service is importing, launching and running the import jobs for the
orders and order-lines files.

\begin{lstlisting}[
    style=java,
    caption=A selection of code from the `ImportJobService' in the application.
    Dots (...) represent skipped code.,
    label={lst:service}]
@Service
@Slf4j
@ToString
public class ImportJobService implements ImportJob {
    private final JobLauncher jobLauncher;

    public ImportJobService(final JobLauncher jobLauncherParameter) {
        jobLauncher = jobLauncherParameter;
    }

    private static String writeFileToTempFile(final MultipartFile multipartFile) {
        ...
    }

    @Override
    public void launchImportJob(final Job job, final MultipartFile multipartFile) {
        ...
    }

    private void runImportJob(final Job job, final String tempFileAbsolutePath) {
        ...
    }
}
\end{lstlisting}
