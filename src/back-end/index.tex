% textidote: ignore begin
\section{Back end}\label{sec:back-end}
% textidote: ignore end

In this section, we will show the features of the back end, hereby showcase the most important functions,
the libraries used and the database integration.

The back end of an application can mean many things and can be of very different complexities.
For our solution, we have implemented a single Spring Boot application containing an HTTP server that handles all
relevant requests and responses.
We therefore only need a single container deployed for the back end.

The backend handles the model and controller part of the MVC architecture and is written in the Java programming
language.
The reasons for this choice are multiple:

\begin{itemize}
    \item Java has existed for a long time, why it has stood the test of time.
    For this reason, Java is also very well documented, which makes it easy to work with.
    \item Secondly, Java is widely used in the industry, why the team members get valuable experience and knowledge when
    working with the language.
    \item Lastly, it is heavily suggested by the university that the group uses the language for their P3 project.
\end{itemize}

% textidote: ignore begin
\subsection{Back-End Framework}\label{subsec:back-end-framework}
% textidote: ignore end

The framework used in the back end is Spring Boot~\cite{springboot2024}.
Spring Boot is an open-source framework for the Java programming language and is built on top of the Spring framework.
The reason for choosing this approach is rather obvious as this framework alone encapsulates all features needed for our
application out of the box.
It provides a relatively easy way to develop a REST API for web applications, as it has built-in support for the MVC
architecture.
In relation to the MVC architecture, Spring Boot is used for the controller and model.
With Spring Boot, the Gradle Build Tool~\cite{gradle2024} for Java is used to handle dependency management, incremental
builds, build caching among other important things.

% textidote: ignore begin
\subsection{Back-end libraries}\label{subsec:back-end-libraries}
% textidote: ignore end

% textidote: ignore begin
\subsection{Important back-end functions}\label{subsec:important-back-end-functions}
% textidote: ignore end

% TODO: Add imporant functions and describe them

% textidote: ignore begin
\subsection{Database}\label{subsec:database}
% textidote: ignore end

The database option used is PostgresSQL~\cite{postgresql2024}, which is an open source relational database.
A relational database was chosen over a non-relational one due to the choice of back end batch processioning of the
imported CSV files.
In this way of processing, the implementation includes several tables in the database.

A table of all the database tables can be seen in Table~\ref{tab:database-tables}.
In this table, it is clear that most database tables regard the batch processing.
The tables `order' and `order\_line' in contrast regards the final storage of user imported CSV files.
A non-relational database could also have been used for these, as the interdependencies of these tables are rather
simple.

\begin{table}[H]
    \centering
    \begin{tabular}{c}
        \toprule
        \textbf{Database tables}
        \\ \midrule
        batch\_job\_execution
        \\ \midrule
        batch\_job\_execution\_context
        \\ \midrule
        batch\_job\_execution\_params
        \\ \midrule
        batch\_job\_instance
        \\ \midrule
        batch\_step\_execution
        \\ \midrule
        batch\_step\_execution\_context
        \\ \midrule
        order
        \\ \midrule
        order\_line
        \\ \bottomrule
    \end{tabular}%
    \caption{List of all database tables.
    }\label{tab:database-tables}
\end{table}

Furthermore, the database is run in its own container.
This is important as it makes potential maintenance easier, which is good practice in general.

% textidote: ignore begin
\subsubsection{Choice of relational database}\label{subsubsec:choice-of-relational-database}
% textidote: ignore end

During development, the PostgreSQL container could have been replaced with
% textidote: ignore begin
an H2
% textidote: ignore end
database~\cite{h22024} for two reasons:

\begin{itemize}
    \item The drawback with the
% textidote: ignore begin
    H2
% textidote: ignore end
    database is that it is an in-memory database.
    This means that the data will be lost on termination of the database instance.
    However, in development, the PostgreSQL Driver dependency was used, which starts and terminates the container
    containing the database on each reload of the back end.
    This results in the same drawback.
    \item The
% textidote: ignore begin
    H2
% textidote: ignore end
    database is all things considered easier and simpler to implement and work with.
    However, since the developers wanted to learn PostgreSQL and the handling thereof and because we would
    need it in the end anyway when deploying, we decided to go with the PostgreSQL database throughout the development.
\end{itemize}

