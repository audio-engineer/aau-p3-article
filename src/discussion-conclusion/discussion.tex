% textidote: ignore begin
\section{Discussion}\label{sec:discussion-conclusion}
% textidote: ignore end

\subsection{Future Development}\label{subsec:future-development}
In this section, we will discuss the potential future development of the system.
The system has a lot of potential for future development, as it is a very basic implementation of the idea.
All the features that could be implemented in the future will be provided in a list below, with an expanded
explanation of each feature.

\begin{itemize}
    \item API Integration
    \item Faster response time
    \item Update an order
\end{itemize}

\noindent
\newline
\textbf{API Integration}

\noindent
The system could be more user-friendly if the system could be integrated with an API instead of uploading CSV files
to the system.
By having an API, the system would seamlessly update the data in the system, and the user would not have to upload,
making the system more automated and user-friendly.
The reason why this was not implemented is because of the~\acrshort{epos} system of the company, which does not provide
an API\@.

\noindent
\textbf{Faster response time}

\noindent
The system could be improved by having a faster response time.
The system is currently already fast enough for the user to use, but it could always be improved.
One of our would-haves was to have a faster response time, but we did have time to prioritize optimizing the system
for speed.
One way to improve the response time could be to optimize the reading and writing of the data in the system.
We could make asynchronous calls to reading the CSV files, which would make the system faster.

\noindent
\newline
\textbf{Update an order}

% todo: This could be moved to the next section and be a part of the software requirements evaluation
\noindent
The system could be improved by having the ability to update an order.
Currently, the system only allows the user to view the orders and then look at the details of the order.
The user cannot update the order, which could be a useful feature if something goes wrong with the order or is not
displayed correctly.
This feature was not implemented as it was not a priority for the systems' functionality, and it was labeled as a
could-have in Section~\ref{subsec:software-requirements-specification}.

\subsection{Software requirements evaluation}\label{subsec:software-requirements-evaluation}

In this section, we will evaluate the software requirements that were set at the start of the project.
We will look at both the tables of functional~\ref{tab:functional-requirements-specification}
and non-functional requirements~\ref{tab:non-functional-requirements-specification}.

\subsubsection{Must-haves}\label{subsubsec:must-haves}

In this section, we will evaluate the must-haves requirements.
Most of our must-haves have been implemented, and the system is functional.
That concerns uploading a csv file, creating a user, deleting a user, creating an order, getting a list of all orders
with a particular product ID and getting a list of all orders on a particular date.
\newline

\noindent
\textbf{Getting a list of all customers}\label{text:getting-a-list-of-all-customers}

\noindent
Although this requirement was a must-have, it has not been implemented.
The reason for this is that the requirement was not prioritized as highly as the other must-haves.
From the data we receive from the~\acrshort{epos} system, we do not have the information about the customers.
Therefore, we did not prioritize this requirement as highly as the other must-haves.
Even though this requirement was not implemented, the system is still functional without it and can be used by the
company. % todo: A little too much yapping here, maybe?
We have discussed this with the company, and they are aware that this requirement was not implemented.
But they are still satisfied with the system and the functionality it provides.

%textidote: ignore begin
\subsubsection{Should-haves}\label{subsubsec:should-haves}
% textidote: ignore end

In this section, we will evaluate the should-haves requirements.
A lot of the should-haves have been implemented, and the system is functional with them.
This section will only cover the requirements that have not been implemented.
\newline

\noindent
\textbf{Delete an order}\label{text:delete-an-order}

\noindent
Deleting an order was a should-have requirement, but it has not been implemented.
Once again, this requirement was not prioritized as highly as some other should-haves.
The reason for this is that it is not at all that important to the company to be able to delete an order.
As the company has an~\acrshort{epos} system that can already delete orders, so for that reason, it was not a priority
for the system to be able to delete orders.

\subsection{Final thoughts}\label{subsec:final-thoughts}

The system developed serves as a robust solution for the company.
Providing essential functionalities that the company can immediately use.
The initial implementation of the system effectively addresses several core requirements, such as handling the
csv files, authenticating users and displaying data.
These must-have features are crucial for the company to form a reliable operational framework, even if certain
lower-priority features were not implemented, like deleting an order.

The project holds significant potential for future development.
With the addition of API integration, the system could be more user-friendly and automated.
Similarly, faster response times could be achieved through methods like asynchronous data processing, which would
improve the user experience.

The system development journey has been a collaborative and iterative process, with essential features implemented
based on the practical needs of the company.
The system has been designed to be scalable and adaptable, with the potential for future development and expansion as
stated in Section~\ref{subsec:future-development}.
