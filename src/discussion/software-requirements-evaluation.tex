% textidote: ignore begin
\subsection{Software requirements evaluation}\label{subsec:software-requirements-evaluation}
% textidote: ignore end

In this section, we will evaluate the software requirements that were set at the start of the project.
We will look at both the tables of functional~\ref{tab:functional-requirements-specification}
and non-functional requirements~\ref{tab:non-functional-requirements-specification}.

\subsubsection{Must-haves}\label{subsubsec:must-haves}

In this section, we will evaluate the must-haves requirements.
Most of our must-haves have been implemented, and the system is functional.
That concerns uploading a csv file, creating a user, deleting a user, creating an order, getting a list of all orders
with a particular product ID and getting a list of all orders on a particular date.

\begin{itemize}
    \item \textbf{Getting a list of all customers:}
    Although this requirement was a must-have, it has not been implemented.
    The reason for this is that the requirement was not prioritized as highly as the other must-haves.
    From the data we receive from the~\acrshort{epos} system, we do not have the information about the customers.
    Therefore, we did not prioritize this requirement as highly as the other must-haves.
    Even though this requirement was not implemented, the system is still functional without it and can be used by the
    company. % todo: A little too much yapping here, maybe?
    We have discussed this with the company, and they are aware that this requirement was not implemented.
    But they are still satisfied with the system and the functionality it provides.
\end{itemize}

%textidote: ignore begin
\subsubsection{Should-haves and could-haves}\label{subsubsec:should-haves-and-could-haves}
% textidote: ignore end

In this section, we will evaluate the should-haves and could-haves requirements.
A lot of the should-haves have been implemented, and the system is functional with them.
This section will only cover the requirements that have not been implemented.

\begin{itemize}
    \item \textbf{Delete an order:}
    Deleting an order was a should-have requirement, but it has not been implemented.
    Once again, this requirement was not prioritized as highly as some other should-haves.
    The reason for this is that it is not at all that important to the company to be able to delete an order.
    As the company has an~\acrshort{epos} system that can already delete orders, so for that reason, it was not a priority
    for the system to be able to delete orders.
    \item \textbf{Update an order:}
    The system could be improved by having the ability to update an order.
    Currently, the system only allows the user to view the orders and then look at the details of the order.
    The user cannot update the order, which could be a useful feature if something goes wrong with the order or is not
    displayed correctly.
    This feature was not implemented as it was not a priority for the systems' functionality, and it was labeled as a
    could-have in Section~\ref{subsec:software-requirements-specification}.
\end{itemize}
