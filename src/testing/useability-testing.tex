% textidote: ignore begin
\subsection{Usability testing}\label{subsec:usability-tests}
% textidote: ignore end

User tests have been conducted, as they allow for testing of how convenient
and useful a function is presented to the user.
For this project, the authors decided to conduct two rounds of user tests.
The first round was planned to spot any major oversights in the design.
If any issues were caught at this point in time, the authors would have approximately one and a half month
to fix the issues, this is also why there was more focus on general functionality instead of details.
Here the authors had already discussed and agreed with Nova on the hard requirements for the project.
It was not possible to conduct this test with people from Nova as the schedules did not line up with the authors.
However, because the hard requirements were decided that it would also be useful to get feedback
from an outside perspective.
The second round of user testing was done with the help from the Nova Café.
%todo: Specify the conditions of the interview/test.

\subsubsection{User test 1}\label{subsubsec:user-test-1}

The first round took place on~\DTMdisplaydate{2024}{11}{1}{-1}.
This test was conducted as a workshop at Aalborg university where other students also from the software major,
both first Semester students and third Semester.
At the workshop, the authors presented Nova and the problem the company is facing, which is described in
Section~\ref{subsec:the-current-situation}.

When a user accesses the Nova dashboard, they were presented with a main page where it was only possible to see
one diagram at a time.
The authors' initial idea was that, as it will be used on an iPad, it made sense to not clutter or compress too
much information at the same time.
The way you would switch between diagrams was to click on arrows besides the diagram and then switch until you had
the correct diagram.
This idea was challenged by multiple people at the workshop, who mentioned that as this was a dashboard, we should
have more information on the main page.
It was also discussed that the arrows to switch between diagrams were not needed and added extra steps to achieve
the main goals of the dashboard.

Another important part of the design that the authors got a lot of feedback on was the way the user would choose a
timeframe for the data shown.
The first iteration of the design involved a grid of days that would show the selected year,
see Figure~\ref{fig:lofi-prototype}.
Although the idea was viable, it became more questionable when taken into account that it was to be used on an iPad.
The main problem with the approach was that it could feel weird or annoying to use when wanting to have specific dates
for the data.
To fix the issues with the grid design, a different solution was agreed on, see Section~\ref{subsec:final-design}.

Overall, the feedback received through the workshop allowed the authors to make major changes to design or
functionality, while still having enough time to finish the product.

% textidote: ignore begin
\subsubsection{User test 2}\label{subsubsec:user-test-2}
% textidote: ignore end

%todo: Get feedback from someone at Nova.
