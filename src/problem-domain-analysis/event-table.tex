% textidote: ignore begin
\subsection{Event Table}\label{subsec:event-table}
% textidote: ignore end

Multiple events were identified in the problem domain system.
An event represents one of the unique behaviors of one of the classes that were presented earlier in
Section~\ref{subsec:classes-objects-and-structure}.

The event table that can be seen below in Table~\ref{tab:event-table} visualizes the connections between all the
classes and events, and uses the ``updated'' even table model, as described by Mathiassen~\cite[102]{mathiassen2018}.

\newcommand{\rot}{\rotatebox{90}}
\newcolumntype{C}{>{\centering\arraybackslash}X}
\newcolumntype{g}{>{\columncolor[HTML]{aaaaaa}}c}

\begin{table}[H]
    \centering
    \begin{tabularx}{\textwidth}{ c p{4.5cm} C C C C C }
        & & \multicolumn{5}{ g }{Class} \\
        & & User & Admin (Owner) & Employee (Barista) & Order & Product
        \\
        \cmidrule{2-7}
        & Employed & & & + & &
        \\
        & Changed & & & \textasteriskcentered{} & &
        \\
        & Terminated & & & + & &
        \\
        & New order data available & & & & + &
        \\
        \rot{\rlap{~Event}}
        & Visualization viewed & \textasteriskcentered{} & \textasteriskcentered{} & \textasteriskcentered{}      & &
        \\
        & Product name mapped & & & & & \textasteriskcentered{}
        \\
        & Order data edited & & & & \textasteriskcentered{} &
        \\
        \cmidrule{2-7}
    \end{tabularx}
    \caption{The system's event table.
    \textbf{+} indicates that the event in question can occur zero or one time.
    \textbf{\textasteriskcentered{}} indicates it can occur zero or more times.}\label{tab:event-table}
\end{table}
