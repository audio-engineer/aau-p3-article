% textidote: ignore begin
\subsection{Back-end libraries}\label{subsec:back-end-libraries}
% textidote: ignore end

In the back end we use several libraries and external code and services to enable and leverage the functionality of our
application.
Some of the most important ones are mentioned in this section.

\subsubsection{Amazon Cognito}

For user sign-in and access control, we use Amazon Cognito~\cite{cognito2024}.
Amazon Cognito provides quality of life features for the sign-up and sign-in process of users.
By using this service, we eliminate the need to implement our own sign-up and sign-in logic and are able to focus on the
core mechanics of our application.
Amazon Cognito seamlessly provides secure access to the website and furthermore enables 2-factor-authentication.
Users and admins can be added in their interface and will then be able to access the NOVA Dashboard.
Importantly, as the application is tailored for NOVA specifically, the full access to the Cognito dashboard can be
granted to the owners of NOVA without issue and further customization.

% textidote: ignore begin
\subsubsection{Spring Boot dependencies}
% textidote: ignore end

Lastly, we want to mention some interesting Spring Boot dependencies used in the back-end application.
These include:

\begin{itemize}
    \item \textbf{MapStruct:}
    This dependency, also called~\url{org.mapstruct:mapstruct}, allows for easier mapping between DTOs and
    entity models.
    In short, it simplifies complex mapping logic.

    \item \textbf{Batch Processing:}
    This dependency, also called~\texttt{spring-boot-starter-batch}, provides an abstraction for batch processing.

    \item \textbf{Spring Boot Starters:}
    We use several starters.
    One of these is the~\texttt{spring-boot-starter-oauth2-resource-server} dependency, which is used for
    OAuth2 authentication.

    \item \textbf{CSV Validation:}
    The~\url{uk.gov.nationalarchives:csv-validator-java-api} dependency is used to validate and process CSV
    files.
\end{itemize}
