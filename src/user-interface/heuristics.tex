\subsection{Heuristics}\label{sec:heuristics}

Throughout the design period, the group has kept a strong focus on Jakob Nielsen's 10 usability
heuristics~\cite{usability-heuristics}.
These are a set of general principles that can be used to evaluate the usability of a user interface.
While the group could not satisfy all the heuristics, many of them were taken into consideration when designing the
user interface.
The following is a list of the most relevant heuristics and how they were implemented in the design.

% textidote: ignore begin
\subsubsection{}\label{subsubsec:visibility-of-system-status}
% textidote: ignore end

This heuristic is about keeping the user informed about what is happening.
It is achieved by showing a spinning wheel when the data is being loaded, and by showing a status notification
when order data is being imported or when settings are being saved.

% textidote: ignore begin
\subsubsection{Consistency and standards}\label{subsubsec:consistency-and-standards}
% textidote: ignore end

This heuristic is about making sure that the design is consistent.
By using the MUI library, explained in Section~\ref{subsec:front-end-libraries}, the group ensures that the design is
consistent.
Consistency is also achieved by using the same color scheme throughout the design.
The navigation bar and the filter options are also consistent across the different pages that support them.

% textidote: ignore begin
\subsubsection{Error prevention}\label{subsubsec:error-prevention}
% textidote: ignore end

This heuristic is about making sure that the user does not make mistakes.
The system has a number of error preventions, such as checks to make sure that the user is uploading the correct file.

% textidote: ignore begin
\subsubsection{Flexibility and efficiency of use}\label{subsubsec:flexibility-and-efficiency-of-use}
% textidote: ignore end

This heuristic is about making sure that the system can be used efficiently.
It is achieved by making the charts interactive and by allowing the user to filter the data.
Furthermore, the user can create categories for the charts, which allows for a more detailed analysis of the data.

% textidote: ignore begin
\subsubsection{Aesthetic and minimalist design}\label{subsubsec:aesthetic-and-minimalist-design}
% textidote: ignore end

This heuristic is about making sure that the design is aesthetically pleasing, by focusing on the content.
It is achieved by bringing the focus to the charts.
The libraries used for the interface and charts are also minimalistic in design and the navigation bar is collapsible,
which creates more space for the charts.
