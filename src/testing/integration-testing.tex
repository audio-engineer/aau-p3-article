% textidote: ignore begin
\subsection{Integration Testing}\label{subsec:integration-tests}
% textidote: ignore end

When programming projects reach a certain size, multiple parts are usually introduced to solve different problems,
that would otherwise be more challenging when only using one language or technology.
For this project, the technology stack that has been chosen is described in Section~\ref{sec:technologies}.
In short, the choice was made to have a Java back end which exposes a REST API for the front end, but also connects to
a PostgreSQL database.
As the back end is the only part that communicates with the SQL database, there was a need to test the connection
between the Java back end and the database, which is what integration testing is for.

\subsubsection{Database save test}\label{subsubsec:database-save-integration-test}

There are many ways to test if the connection is configured correctly, but a straightforward approach involves trying
to write to the database and then asserting that the operation was a success.

\begin{lstlisting}[
    style=java,
    label={lst:database-order-test},
    caption={A code snippet of the database order model test.
    Dots (...) represent skipped code.},
    captionpos=b,
]
@Test
final void orderSaveTest() {
  ...
  orderRepository.save(ORDER);
}
\end{lstlisting}

In Listing~\ref{lst:database-order-test} is shown one of the simpler tests.
The test relies on the fact that the save method can fail by throwing an exception, which would implicitly fail the
test.

% textidote: ignore begin
\subsubsection{Persistence test}\label{subsubsec:persistence-integration-test}
% textidote: ignore end

\begin{lstlisting}[
    style=java,
    label={lst:database-persistence-test},
    caption={A code snippet of a persistence test},
    captionpos=b,
]
final void orderFindTest() {
  entityManager.persist(ORDER);

  final Order firstFetched = entityManager.find(Order.class, ORDER.getId());

  final Order secondFetched = entityManager.find(Order.class, ORDER.getId());

  assertThat(firstFetched).isEqualTo(secondFetched);
}
\end{lstlisting}

Retrieving from the database is also critical functionality, which is why tests have also been made to verify it works.
As seen in Listing~\ref{lst:database-persistence-test}, the test persists an Order to the database through the
entityManager.
