\subsection{Unit Testing}\label{subsec:unit-tests}

The authors have chosen that for all code, if a test is made, it must pass before some functionality can be
deemed completed.
This means that it is common to all tests that they must pass when code is pushed to our main branch.
Therefore, we have a 100\% pass rate for the tests made, but this does not mean that we
have 100\% test coverage, as that was deemed outside the scope of this project.

\begin{lstlisting}[
    style=java,
    label={lst:aaa-test},
    caption={Pseudocode that shows the~\acrshort{aaa} pattern for testing.},
    captionpos=b,
]
test() {
    // Arrange
    SomeState someState = new SomeState(foo);

    // Act
    someState.changeState(bar);

    // Assert
    assert(someState == expectedState);
}
\end{lstlisting}

As seen in Listing~\ref{lst:aaa-test}, to use the~\acrshort{aaa} pattern,
one would first set some state to a known state.
Then you act on the state and transform it through some function, also called unit.
We can finally assert that we get the expected state from the transformation, assert can be seen
as the test passing if the expression returns true, and failing if false.

The software solution developed for this project is based on the~\acrfull{mvc} architecture.
Both the model and controller are part of the backend, and because the backend is a Java program,
the spring-framework is used.
For testing of our view, which is our frontend build as a node project, we have tested through usability tests,
see Section~\ref{subsec:usability-tests}.
By using Spring, we also get many high quality tools to test the code through Spring Test with JUnit.

\subsubsection{Testing fromString}\label{subsubsec:fromstring-unit-test}

When we are parsing the csv file, an important step is to convert the string value of the field into the desired
datatype.
For some data types, this conversion happens automatically, but for custom types the authors need to implement the
fromString method, as a csv file is stored as text.
One example of this implementation is for both the OrderStatus and PaymentStatus.
These two types are enums, and as they are both implemented the same way, it makes sense to only show how one of the
enums work.
OrderStatus has been chosen as the example.
The OrderStatus enum has two valid values~\(ARCHIVED\) and~\(DELETED\).
It would then make sense to test the conversion for the valid values and at least one invalid value to check that
invalid data is not accepted.

\begin{lstlisting}[
    style=java,
    label={lst:enum-convertion-order_status},
    caption={Testing the valid and an invalid OrderStatus conversion from a string.},
    captionpos=b,
]
final void testOrderStatusArchivedFromString() {
  final OrderStatus orderStatus = OrderStatus.fromString("ARCHIVED");

  assertThat(orderStatus).isEqualTo(OrderStatus.ARCHIVED);
}

final void testOrderStatusDeletedFromString() {
  final OrderStatus orderStatus = OrderStatus.fromString("DELETED");

  assertThat(orderStatus).isEqualTo(OrderStatus.DELETED);
}

@SuppressWarnings("TestMethodWithoutAssertion")
final void testInvalidOrderStatus() {
  final String invalidOrderStatus = "CREATED";

  assertThatThrownBy(() -> OrderStatus.fromString(invalidOrderStatus))
      .isInstanceOf(UnknownStatusException.class)
      .hasMessage("Unknown order status: %s", invalidOrderStatus);
}
\end{lstlisting}

In the code snippet~\ref{lst:enum-convertion-order_status}, the code implicitly follows the~\acrshort{aaa} pattern,
where an OrderStatus is arranged, it is then acted on, by converting from a string.
Finally, the tests assert that either the correct enum has been generated or an exception has been thrown for invalid
strings.

\subsubsection{Testing lazy loading}\label{subsubsec:lazy-loading-unit-test}

An interesting feature of the database setup is lazy loading, which is when the program only retrieves an object from
the database when it is necessary.
This is useful to not waste resources communicating with the database unless it is necessary.

\begin{lstlisting}[
    style=java,
    label={lst:test-lazy-loading},
    caption={Testing that an object from the database is not retrieved when not directly accessed.},
    captionpos=b,
]
final void lazyAssociationsTest() {
  final Order order = orderRepository.findById(ORDER_ID).orElseThrow();

  assertThat(persistenceUnitUtil.isLoaded(order.getOrderLines())).isFalse();
}
\end{lstlisting}

As seen in Snippet~\ref{lst:test-lazy-loading}, the test ensures that lazy loading is active, because the order was not
accessed, but only referenced.

\subsubsection{Testing the controller}\label{subsubsec:controller-unit-test}
Another more complex test case the authors also deemed essential is the controller test, specifically when
the backend receives a POST request on the /api/orders endpoint.
The endpoint is used to upload order data as a CSV file, for more information as explained in
Chapter~\ref{ch:implementation}.
The goal of the endpoint is to populate the database with the order data, and to do this safely we need
to make sure that all the data received is valid data for the Order model.
This is the reason that we created a validator, and to test it we also created the test in
Listing~\ref{lst:order-controller-test}.

\begin{lstlisting}[
    style=java,
    label={lst:order-controller-test},
    caption={Order Controller test.},
    captionpos=b,
]
class OrderControllerTests {
  @Value("classpath:orders.csv")
  private Resource resource;

  final void testUploadValidFileShouldReturnOk() throws Exception {
    final byte[] ordersCsvFile = Files.readAllBytes(resource.getFile().toPath());

    final MockMultipartFile mockMultipartFile =
        new MockMultipartFile(
            "orders",
            SupportedFiles.ORDERS_CSV.getFileNamePrefix(),
            "text/csv",
            ordersCsvFile
        );

    final MockHttpServletResponse response =
        mockMvc
            .perform(
                MockMvcRequestBuilders.multipart("/api/orders")
                    .file(mockMultipartFile)
                    .contentType(MediaType.MULTIPART_FORM_DATA))
            .andReturn()
            .getResponse();

    assertThat(response.getStatus()).isEqualTo(200);
  }
}
\end{lstlisting}

In this test, we make use of Spring utilities for creating and performing HTTP request in a test
environment.
This test receives the mockMultiPartFile, which is a valid CSV file, and it is then expected that the
upload should succeed and send an HTTP status code which indicates success,
which is also known as status code 200.
