\subsection{Functions}\label{subsec:functions}

In this section we will analyze the functions of the system.
There are four different function types, hereby Update, Signal, Read and Compute~\cite{mathiassen2018}.
In short, the different function types have the following meanings:

\begin{itemize}
    \item \textbf{Update} functions are initiated by a problem domain event and changes state.
    \item \textbf{Signal} functions are initiated by a change in the state of a model and react accordingly.
    \item \textbf{Read} functions are initiated by an actor and retrieves and displays relevant information about a
    model.
    \item \textbf{Compute} functions are initiated by an actor when there is a need for information processing, data
    transformation or calculation and displays the result.
\end{itemize}

There are also four different complexities, hereby Simple, Medium, Complex and Very complex~\cite{mathiassen2018}.
In short, the different function complexities have the following meanings:

\begin{itemize}
    \item \textbf{Simple} functions perform straightforward operations on a single attribute of an object.
    \item \textbf{Medium} functions create and manipulate single objects.
    \item \textbf{Complex} functions involve interaction between multiple objects.
    They may also have conditional logic.
    \item \textbf{Very complex} functions have several goals and involve several objects and processes.
    They also have branching and multiple decision capabilities among others.
\end{itemize}

In Table~\ref{tab:functions} are all non-trivial functions outlined.
For clarity, we define trivial functions to be basic utility functions and one-line functions, such as getter-functions.
All of these functions are derived from a combination of the use-cases section in Section~\ref{subsec:use-cases},
the section about actors in Section~\ref{subsec:actors}, the event table in Section~\ref{subsec:event-table} and the
section about classes in Section~\ref{subsec:classes-objects-and-structure}.

\begin{table}[h]
    \centering
    \begin{tabular}{ccc}
        \toprule
        \textbf{Function name}
        & \textbf{Type}
        & \textbf{Complexity}
        \\ \midrule
        Find user information
        & Read
        & Simple
        \\ \midrule
        Find products by category
        & Read
        & Medium
        \\ \midrule
        Create user session
        & Update
        & Medium
        \\ \midrule
        Validate user session
        & Compute
        & Medium
        \\ \midrule
        Format data
        & Compute
        & Complex
        \\ \midrule
        Validate and process CSV file
        & Compute
        & Very complex
        \\ \bottomrule
    \end{tabular}%
    \caption{Table of all functions and their respective assigned types and complexities.
    Trivial functions are excluded.
    }\label{tab:functions}
\end{table}

The~\textit{Validate and process CSV file} function is classified as very complex due to the number of different
sub-functions it contains.
In Table~\ref{tab:functions-validate-and-process-csv-file} all non-trivial sub-functions regarding
the~\textit{Validate and process CSV file} function are further clarified.

\begin{table}[h]
    \centering
    \begin{tabular}{ccc}
        \toprule
        \textbf{Function name}
        & \textbf{Type}
        & \textbf{Complexity}
        \\ \midrule
        Validate CSV file type
        & Read
        & Simple
        \\ \midrule
        Handle exceptions
        & Signal
        & Simple
        \\ \midrule
        Save CSV contents
        & Update
        & Medium
        \\ \midrule
        Create additional meta-attributes
        & Update
        & Medium
        \\ \midrule
        Match CSV field to model
        & Compute
        & Medium
        \\ \midrule
        Check for duplicates
        & Compute
        & Medium
        \\ \bottomrule
    \end{tabular}%
    \caption{Table of all sub-functions and their respective type and complexity regarding the very complex parent
    function~\textit{Validate and process CSV file} seen in Table~\ref{tab:functions}.
    Trivial functions are excluded.
    }\label{tab:functions-validate-and-process-csv-file}
\end{table}
