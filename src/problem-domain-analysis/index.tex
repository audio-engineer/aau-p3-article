% textidote: ignore begin
\chapter{Problem Domain Analysis}\label{ch:problem-domain-analysis}
% textidote: ignore end

The following chapter will deal with the analysis and understanding of the problem domain.
The aim of such an analysis is to answer the question:
\textcquote[47]{mathiassen2018}{What information should the system deal with?}
The answer to this question is crucial because the information that the future system will deal with decides the
requirements and building blocks of the system.
To find the answer to this question, we will first model the classes, objects and structures of the system, and the
relationships and interaction between them.
Such an analysis follows the core principle of
\textcquote[47]{mathiassen2018}{[model] the real world as future users will see it.}

First, in Section~\ref{sec:classes-objects-and-structure}, we will identify the classes, objects and structures between
those classes and objects.
A class diagram will clarify the connections between the individual elements, to depict their role in the problem domain
system.

In Section~\ref{sec:event-table} of this chapter, we will use an event table~\cite[52]{mathiassen2018} to describe and
categorize the activities of specific class objects and how they interact with each other.

Overall, the goal of the class diagram and the event table is to visualize the problem domain in a clear and unambiguous
manner.
The classes, objects and events that will be identified here will be evaluated by the criteria that they
\textcquote[63]{mathiassen2018}{should refer to phenomena that future users will administrate, monitor, or control in
their work.}
The two visualizations will serve as a basis for all further analyses, and will play a crucial role later on in the
design and implementation phase of the project.

% textidote: ignore begin
\section{Classes, Objects and Structure}\label{sec:classes-objects-and-structure}
% textidote: ignore end

\section{Event Table}\label{sec:event-table}

