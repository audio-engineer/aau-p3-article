% textidote: ignore begin
\section{Software Requirements Specification}\label{sec:software-requirements-specification}
% textidote: ignore end

For further clarification and alignment of features to prioritize the MoSCoW model is employed~\cite{hudaib2018}.
Herein an overview of the requirements for the software is covered.
\textbf{``Must-haves''} (Mo) include features that must be included in the software.
\textbf{``Should-haves''} (S) include features that are desired and possible to implement, but are not prioritized
due to time limitations.
\textbf{``Could-haves''} (Co) include features that are possible and valuable to implement but are not the main
focus of the report.
\textbf{``Won't-haves''} (W) include features that are out of scope.

Furthermore, the requirements are split into two different tables, functional- and non-functional requirements.
Functional requirements describe what actions the system must be capable of performing.
Non-functional requirements, however, describe how the system must function~\cite{benyon2019}.
This is done among other reasons to distinguish between functionality and quality attributes.
By having these separated into different tables, it is easier to plan design decisions and maintain a clear overview.

% textidote: ignore begin
\begin{table}[H]
    \centering
    \resizebox{\columnwidth}{!}{%
        \begin{tabular}{ccc}
            \toprule
            \textbf{Functional requirement}
            & \textbf{MoSCoW annotation}
            & \textbf{Motivation}
            \\ \midrule
            Functional requirement placeholder
            & Mo
            & {}
            \\ \midrule
            Functional requirement placeholder
            & S
            & {}
            \\ \midrule
            Functional requirement placeholder
            & Co
            & {}
            \\ \midrule
            Functional requirement placeholder
            & W
            & {}
            \\ \bottomrule
        \end{tabular}%
    }
    \caption{MoSCoW model of the functional requirements specification.
    The \textit{Functional requirement} column describes the listed requirements for the software.
    The \textit{MoSCoW annotation} column covers the category to which the requirements belong.
    The \textit{Motivation} column encompasses where in the report the motivation for the requirements is located.
    }\label{tab:functional-requirements-specification}
\end{table}
% textidote: ignore end


% textidote: ignore begin
\begin{table}[H]
    \centering
    \resizebox{\columnwidth}{!}{%
        \begin{tabular}{ccc}
            \toprule
            \textbf{Non-functional requirement}
            & \textbf{MoSCoW annotation}
            & \textbf{Motivation}
            \\ \midrule
            Non-functional requirement placeholder
            & Mo
            & {}
            \\ \midrule
            Non-functional requirement placeholder
            & S
            & {}
            \\ \midrule
            Non-functional requirement placeholder
            & Co
            & {}
            \\ \midrule
            Non-functional requirement placeholder
            & W
            & {}
            \\ \bottomrule
        \end{tabular}%
    }
    \caption{MoSCoW model of the non-functional requirements specification.
    The \textit{Non-functional requirement} column describes the listed requirements for the software.
    The \textit{MoSCoW annotation} column covers the category to which the requirements belong.
    The \textit{Motivation} column encompasses where in the report the motivation for the requirements is located.
    }\label{tab:non-functional-requirements-specification}
\end{table}
% textidote: ignore end
