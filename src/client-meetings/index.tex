At the beginning of the project, the authors interviewed the owners of NOVA to get an understanding of the problem at
hand.
This interview was not recorded, as it was meant to introduce the authors and the owners of NOVA, while brainstorming
ideas, ending on the idea of an analytics app.
After researching current solutions on analytics, another meeting was set up to get more specific ideas and decide
on the approach.
The interview was conducted on~\DTMdisplaydate{2024}{9}{24}{-1}.
The form of the interview was an informal interview where the owners were asked a series of questions and below is
the summarized questions and answers for the meeting.

\begin{itemize}
    \item What~\acrfull{epos} do you currently use, and what are the issues with the system?
    \begin{itemize}
        \item We currently use an of-the-shelf system, that has general analytics integrated into itself.
        \item Because the system is not made with analytics in mind, we do not have access to hourly analytics.
        The data has a specific time associated, but we cannot make use of it through the~\acrshort{epos}.
    \end{itemize}

    \item What are your thoughts on existing analytics software like Google Analytics or Power BI\@?
    \begin{itemize}
        \item These analytics tools are powerful enough for all our use cases, but we do not have the necessary
        knowledge to use them optimally.
        \item We would want something that is straightforward, so that the time investment
        required to use the system makes sense.
    \end{itemize}

    \item How do you picture the solution to your analytics requirements?
    \begin{itemize}
        \item As we use an iPad as our current~\acrshort{epos}, it would be nice if the analytics were also made for
        iPad usage.
        \item An app would be great, as it would be the simplest to use.
        \item We could also make use of a web app if this is simpler to make.
        \item Because we are only two owners, who also function as administrators.
        We need to be able to add the baristas, so they can upload the data as well.
    \end{itemize}

    \item How can we get the data for analysis from the~\acrshort{epos}?
    \begin{itemize}
        \item Because the~\acrshort{epos} was made by a startup with limited resources, it is not possible to access the
        data through a public API, and the only way we can retrieve the data is to export it as a CSV file.
        \item Exporting this data is a manual procedure, and the only way to upload the data to the proposed solution
        would be through a file upload.
    \end{itemize}
\end{itemize}
