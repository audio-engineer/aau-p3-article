% textidote: ignore begin
\section{Back End}\label{sec:back-end}
% textidote: ignore end

This section showcases the features of the back end, hereby the most important functions, the libraries used
and the database integration.

The back end of an application can mean many things and can be of very different complexities.
The back end consists of a single Spring Boot application containing an HTTP server that handles all
relevant requests and responses.
Therefore, only a single container deployed for the back end is needed.

The back end handles the model and controller part of the MVC architecture and is written in the Java programming
language.
The reasons for this choice are multiple:

\begin{itemize}
    \item Java has existed for a long time, why it has stood the test of time.
    For this reason, Java is also very well documented, which makes it easy to work with.

    \item Secondly, Java is widely used in the industry, why the team members get valuable experience and knowledge when
    working with the language.

    \item Lastly, it is heavily suggested by the university that the group uses the language for their P3 project.
\end{itemize}

% textidote: ignore begin
\subsection{Back-End Framework}\label{subsec:back-end-framework}
% textidote: ignore end

The framework used in the back end is Spring Boot~\cite{springboot2024}.
Spring Boot is an open-source framework for the Java programming language and is built on top of the Spring framework.
The reason for choosing this approach is rather obvious as this framework alone encapsulates all features needed for our
application out of the box.
It provides a relatively easy way to develop a REST API for web applications, as it has built-in support for the MVC
architecture.
In relation to the MVC architecture, we use Spring Boot for the controller and model.
With Spring Boot, we use the Gradle Build Tool~\cite{gradle2024} for Java to handle dependency management, incremental
builds, build caching among other important things.

% textidote: ignore begin
\subsection{Front-End Libraries}\label{subsec:front-end-libraries}
% textidote: ignore end

The group uses several front-end libraries to aid in the development of the UX and UI of the application.
The use of these libraries keeps the scope of the project focused on the core functionality of the application, allowing
the group to deliver a product that efficiently meets the requirements of the project.

% textidote: ignore begin
\subsubsection{Material-UI}\label{subsubsec:material-ui}
% textidote: ignore end

To make the user interface appealing, the group uses Material-UI, a React component library that implements Google's
Material Design~\cite{material-ui}.
It provides a set of components that follow the Material Design guidelines, which makes it easy to create a consistent
and visually appealing user interface.
This library is used to create the core components of the application, such as buttons, text fields, and data grids.

% textidote: ignore begin
\subsubsection{Nivo}\label{subsubsec:nivo}
% textidote: ignore end

Nivo is a React component library that provides a set of customizable chart components~\cite{nivo2024}.
From the design phase, the group agreed to use specific types of charts to visualize the data in the application.
There are a lot of React libraries that provide chart components, but many of them did not meet the group's requirements.
They were either missing important charts or had subpar UX or UI.\@
However, Nivo met the group's requirements as it provided a set of customizable chart components that were
responsive, interactive and visually appealing.
The charts were easy to implement and fit in with the Material Design that the application was following.

% textidote: ignore begin
\subsection{Important Front-End Functions}\label{subsec:important-front-end-functions}
% textidote: ignore end

When discussing the front-end functions of the application, there are several key functions that are important to the
functionality.
The focus in this section is on the functions related to the charting and data visualization aspects of the application.
That includes the Nivo library components and the data fetching functions.

% textidote: ignore begin
\subsubsection{Charting functions}\label{subsubsec:charting-functions}
% textidote: ignore end

The Nivo library provides a number of options that allows the group to customize the charts to fit the application's
design.
Listing~\ref{lst:heatmap-chart} shows one of the components that the group uses to visualize the data, a heatmap.
Any component can adapt to the layout of the application and the data that is being visualized.
The only required prop is the data prop, which is an array of objects that contain the data to be visualized.
Everything else is optional and can be customized to fit the application's design.
The listing shows the customization done to the heatmap component, with custom margins, padding, colors, legend and
tooltip.
The design can be seen on Figure~\ref{fig:design-final-dash1} in Section~\ref{subsec:final-design}.
The rest of the components are similar in customization, however some options may differ depending on the component.

\begin{lstlisting}[
    style=tsxAndTypescript,
    caption=Chart visualization in the Heatmap component.
    Dots (...) represent skipped code.,
    label={lst:heatmap-chart}]
const DailyHourlySales = (): ReactElement => {
  ...
  return (
    <ResponsiveHeatMap
      data={data}
      margin={{ top: 60, right: 90, bottom: 60, left: 90 }}
      forceSquare={false}
      xInnerPadding={0.05}
      yInnerPadding={0.05}
      borderRadius={6}
      axisTop={{
        tickSize: 5,
        tickPadding: 5,
      }}
      axisLeft={{
        tickSize: 5,
        tickPadding: 5,
      }}
      colors={{
        type: "sequential",
        scheme: "blues",
      }}
      emptyColor="#555555"
      inactiveOpacity={1}
      legends={[
        {
          anchor: "bottom",
          translateX: 0,
          translateY: 30,
          length: 400,
          thickness: 8,
          direction: "row",
          tickPosition: "after",
          tickSize: 5,
          tickSpacing: 4,
          tickOverlap: false,
          titleAlign: "start",
          titleOffset: 4,
        },
      ]}
      tooltip={({ cell }: TooltipProps<HeatMapDatum>) => {
        return (
          <div
            style={{
              background: "white",
              padding: "9px 12px",
              border: "1px solid #ccc",
            }}
          >
            <div>
              {cell.formattedValue} sale(s) on {cell.serieId}s at {cell.data.x}
            </div>
          </div>
        );
      }}
    />
  );
};
\end{lstlisting}

% textidote: ignore begin
\subsubsection{Data fetching functions}\label{subsubsec:data-fetching-functions}
% textidote: ignore end

The data prop needed for the charting components is fetched from the back end using a REST query.
Listing~\ref{lst:heatmap-data} shows how the data is fetched for the heatmap component.
The data is fetched using the~\texttt{useQuery} hook from the React Query library.
The query key is an array containing the query name and the formatted date range.
The query function is an asynchronous function that fetches the data from the back end.
That data is then transformed using the select function to fit the specific format that's needed for the heatmap
component.

\begin{lstlisting}[
    style=tsxAndTypescript,
    caption=Data fetching in the Heatmap component.
    Dots (...) represent skipped code.,
    label={lst:heatmap-data}]
const DailyHourlySales = (): ReactElement => {
  const { formattedDateRange } = useDateRange();

  const { data } = useQuery({
    queryKey: ["dailyHourlySales", formattedDateRange],
    queryFn: async () => getDailyHourlySales(formattedDateRange),
    enabled: null !== formattedDateRange.from || null !== formattedDateRange.to,
    select: transformToHeatmapData,
  });
  ...
\end{lstlisting}

Listing~\ref{lst:data-fetching} shows the function that fetches the data from the back end.
It creates a query string with the start and end date and sends a GET request to the back end.
The response is returned as a promise.

\begin{lstlisting}[
    style=tsxAndTypescript,
    caption=GetDailyHourlySales function.,
    label={lst:data-fetching}]
const getDailyHourlySales = async (
  formattedDateRange: FormattedDateRange,
): Promise<PaginatedResponse<EmbeddedDailyHourlySales>> => {
  const { from, to } = formattedDateRange;

  const response = await client.get<
    PaginatedResponse<EmbeddedDailyHourlySales>
  >(`/api/daily-hourly-sales?startDate=${from}&endDate=${to}`);

  return response.data;
};
\end{lstlisting}

Listing~\ref{lst:data-transofrm} shows the function that transforms the data from the back end to the format required
for the heatmap component.
The function takes the response from the back end and transforms it into an array of objects.
Each object contains an ID and an array of data points.
The data points are objects that contain the x and y values for the heatmap.
They are sorted by the x value, and the objects are sorted by the ID value.
The function then returns the array of objects.

\begin{lstlisting}[
    style=tsxAndTypescript,
    caption=TransformToHeatmapData function.,
    label={lst:data-transofrm}]
const transformToHeatmapData = (
  response: PaginatedResponse<EmbeddedDailyHourlySales>,
): HeatMapSerie<HeatMapDatum, object>[] => {
  const array = response._embedded.dailyHourlySalesDtoes.map((day) => {
    const data = day.hourlySales.map((hourly) => ({
      x: getLocalizedTime(hourly.hour),
      y: hourly.totalSales,
    }));

    data.sort(
      (
        firstElement: Readonly<HeatMapDatum>,
        secondElement: Readonly<HeatMapDatum>,
      ) => {
        return (
          dayjs(firstElement.x, "HH:mm").unix() -
          dayjs(secondElement.x, "HH:mm").unix()
        );
      },
    );

    return {
      id: getLocalizedDate(day.date),
      data,
    };
  });

  array.sort(
    (
      firstElement: HeatMapSerie<HeatMapDatum, object>,
      secondElement: HeatMapSerie<HeatMapDatum, object>,
    ) => {
      return (
        dayjs(firstElement.id, "DD.MM.YYYY").unix() -
        dayjs(secondElement.id, "DD.MM.YYYY").unix()
      );
    },
  );

  return array;
};
\end{lstlisting}

% textidote: ignore begin
\subsection{Database}\label{subsec:database}
% textidote: ignore end

The database option used is PostgresSQL~\cite{postgresql2024}, which is an open source relational database.
A relational database was chosen over a non-relational one due to the choice of back end batch processioning of the
imported CSV files.
In this way of processing, the implementation includes several tables in the database.

A table of all the database tables can be seen in Table~\ref{tab:database-tables}.
In this table, it is clear that most database tables regard the batch processing.
The tables `order' and `order\_line' in contrast regards the final storage of user imported CSV files.
A non-relational database could also have been used for these, as the interdependencies of these tables are rather
simple.

\begin{table}[H]
    \centering
    \begin{tabular}{c}
        \toprule
        \textbf{Database tables}
        \\ \midrule
        batch\_job\_execution
        \\ \midrule
        batch\_job\_execution\_context
        \\ \midrule
        batch\_job\_execution\_params
        \\ \midrule
        batch\_job\_instance
        \\ \midrule
        batch\_step\_execution
        \\ \midrule
        batch\_step\_execution\_context
        \\ \midrule
        order
        \\ \midrule
        order\_line
        \\ \bottomrule
    \end{tabular}%
    \caption{List of all database tables.
    }\label{tab:database-tables}
\end{table}

Furthermore, the database is run in its own container.
This is important as it makes potential maintenance easier, which is good practice in general.

% textidote: ignore begin
\subsubsection{Choice of relational database}\label{subsubsec:choice-of-relational-database}
% textidote: ignore end

During development, the PostgreSQL container could have been replaced with
% textidote: ignore begin
an H2
% textidote: ignore end
database~\cite{h22024} for two reasons:

\begin{itemize}
    \item The drawback with the
% textidote: ignore begin
    H2
% textidote: ignore end
    database is that it is an in-memory database.
    This means that the data will be lost on termination of the database instance.
    However, in development, the PostgreSQL Driver dependency was used, which starts and terminates the container
    containing the database on each reload of the back end.
    This results in the same drawback.
    \item The
% textidote: ignore begin
    H2
% textidote: ignore end
    database is all things considered easier and simpler to implement and work with.
    However, since the developers wanted to learn PostgreSQL and the handling thereof and because we would
    need it in the end anyway when deploying, we decided to go with the PostgreSQL database throughout the development.
\end{itemize}

