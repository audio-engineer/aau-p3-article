\subsection{Software Requirements Specification}\label{subsec:software-requirements-specification}

For further clarification and alignment of features to prioritize the MoSCoW model is employed~\cite{hudaib2018}.
Herein an overview of the requirements for the software is covered.
\textbf{``Must-haves''} (Mo) include features that must be included in the software for it to be functional.
\textbf{``Should-haves''} (S) include features that are desired and possible to implement, but are not prioritized as
highly as must-haves due to time limitations.
\textbf{``Could-haves''} (Co) include features that are possible and valuable to implement but are not the main
focus of the report.
\textbf{``Won't-haves''} (W) include features that are out of scope and won't be implemented.

Furthermore, the requirements are split into two different tables, functional- and non-functional requirements as seen
in Table~\ref{tab:functional-requirements-specification} and Table~\ref{tab:non-functional-requirements-specification}
respectively.
Functional requirements describe what actions the system must be capable of performing.
Non-functional requirements, however, describe how the system must perform~\cite{benyon2019}.
This is done among other reasons to distinguish between functionality and quality attributes.
By having these separated into different tables, it is easier to plan design decisions and maintain a clear overview.

\begin{table}[H]
    \begin{tabularx}{\textwidth}{ l X X }
        \toprule
        \textbf{Functional requirement}
        & \textbf{MoSCoW annotation}
        & \textbf{Motivation}
        \\ \midrule
        Upload a~.csv-file
        & Mo
        & {\ul{\ref{subsec:cultivating-new-ideas}}}
        \\ \midrule
        Create a user
        & Mo
        & {\ul{\ref{subsubsec:classes-and-objects}}}
        \\ \midrule
        Delete a user
        & Mo
        & {\ul{\ref{subsubsec:classes-and-objects}}}
        \\ \midrule
        Create an order
        & Mo
        & {\ul{\ref{subsubsec:classes-and-objects}}}
        \\ \midrule
        Get a list of all orders with a particular product ID
        & Mo
        & {\ul{\ref{subsec:event-table}}}
        \\ \midrule
        Get a list of all orders on a particular date
        & Mo
        & {\ul{\ref{subsec:event-table}}}
        \\ \midrule
        Get a list of all customers
        & Mo
        & {\ul{\ref{subsec:event-table}}}
        \\ \midrule
        Delete an order
        & S
        & {\ul{\ref{subsubsec:classes-and-objects}}}
        \\ \midrule
        Update a user
        & S
        & {\ul{\ref{subsubsec:classes-and-objects}}}
        \\ \midrule
        Update an order
        & S
        & {\ul{\ref{subsubsec:classes-and-objects}}}
        \\ \midrule
        Export of data to an external file (e.g.\ CSV, JSON, etc.)
        & W
        & {\ul{\ref{subsec:factor-criterion}}}
        \\ \bottomrule
    \end{tabularx}
    \caption{MoSCoW model of the functional requirements specification.
    The \textit{Functional requirement} column describes the listed requirements for the software.
    The \textit{MoSCoW annotation} column covers the category to which the requirements belong.
    The \textit{Motivation} column encompasses where in the report the motivation for the requirements is located.
    }\label{tab:functional-requirements-specification}
\end{table}

\begin{table}[H]
    \begin{tabularx}{\textwidth}{ l X X }
        \toprule
        \textbf{Non-functional requirement}
        & \textbf{MoSCoW annotation}
        & \textbf{Motivation}
        \\ \midrule
        Data visualizations (graphs and charts)
        & Mo
        & {\ul{\ref{subsec:cultivating-new-ideas}}}
        \\ \midrule
        Usability in a tablet browser
        & Mo
        & {\ul{\ref{subsec:cultivating-new-ideas}}}
        \\ \midrule
        Data processing (raw data filtering and validation)
        & Mo
        & {\ul{\ref{subsec:system-definition}}}
        \\ \midrule
        A database
        & Mo
        & {\ul{\ref{subsec:system-definition}}}
        \\ \midrule
        The ability to handle a small number of concurrent users
        & Mo
        & {\ul{\ref{subsec:system-definition}}}
        \\ \midrule
        An intuitive and simplistic front-end design
        & S
        & {\ul{\ref{subsec:cultivating-new-ideas}}}
        \\ \midrule
        Mapping product ID to custom product name
        & S
        & {\ul{\ref{subsec:cultivating-new-ideas}}}
        \\ \midrule
        A well-structured codebase
        & S
        & {\ul{\ref{subsec:system-definition}}}
        \\ \midrule
        A fast and efficient frontend framework
        & S
        & {\ul{\ref{subsec:system-definition}}}
        \\ \midrule
        A robust backend framework
        & S
        & {\ul{\ref{subsec:system-definition}}}
        \\ \midrule
        Authentication via login
        & S
        & {\ul{\ref{subsec:system-definition}}}
        \\ \midrule
        Authorization of users
        & S
        & {\ul{\ref{subsec:system-definition}}}
        \\ \midrule
        Machine learning capabilities
        & Co
        & {\ul{\ref{subsec:system-definition}}}
        \\ \midrule
        A low response time
        & Co
        & {\ul{\ref{subsec:system-definition}}}
        \\ \midrule
        User-interactive data visualizations
        & W
        & {\ul{\ref{subsec:system-definition}}}
        \\ \midrule
        Other security measures (encryption)
        & W
        & {\ul{\ref{subsec:system-definition}}}
        \\ \midrule
        Scalability
        & W
        & {\ul{\ref{subsec:system-definition}}}
        \\ \midrule
        UI related semantic settings (color, language)
        & W
        & {\ul{\ref{subsec:system-definition}}}
        \\ \bottomrule
    \end{tabularx}
    \caption{MoSCoW model of the non-functional requirements specification.
    The \textit{Non-functional requirement} column describes the listed requirements for the software.
    The \textit{MoSCoW annotation} column covers the category to which the requirements belong.
    The \textit{Motivation} column encompasses where in the report the motivation for the requirements is located.
    }\label{tab:non-functional-requirements-specification}
\end{table}
