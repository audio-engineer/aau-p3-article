% textidote: ignore begin
\subsection{Back-end framework}\label{subsec:back-end-framework}
% textidote: ignore end

The framework used in the back end is Spring Boot~\cite{springboot2024}.
Spring Boot is an open-source framework for the Java programming language and is built on top of the Spring framework.
The reason for choosing this approach is rather obvious as this framework alone encapsulates all features needed for our
application out of the box.
It provides a relatively easy way to develop a REST API for web applications, as it has built-in support for the MVC
architecture.
In relation to the MVC architecture, we use Spring Boot for the controller and model.

Furthermore, Spring Boot supports features such as repositories, services, batch-processing and more.
These are described in the following subsections.

\subsubsection{Controller}
A controller in Spring is a class denoted by having the~\texttt{@Controller} annotation.
A controller is the first point contact when engaging with the REST API\@.
In the NOVA Dashboard application, there is only one explicit controller with two post-mappings for the order-file and
order-lines-file respectively, as seen in Listing~\ref{lst:controller}.
The controller handles all HTTP POST requests to the URLs~\url{/api/orders} and~\url{/api/order-lines}.
It expects files and returns a~\texttt{ResponseEntity} object on success.
In the listing is only included the route for the order-files as they both are handled almost identically by the
controller.

\begin{lstlisting}[style=java, caption=A selection of code from the main controller in the application.
Dots (...) represent skipped code.,label={lst:controller}]
@RestController
@RequestMapping("/api")
...
public class OrderController {
   ...
   @PostMapping(value = "/orders", consumes = MediaType.MULTIPART_FORM_DATA_VALUE)
   public final ResponseEntity<String> orderHandler(
      @Valid @RequestParam("orders") final MultipartFile multipartFile) throws ValidationException {
         ...
   }
   ...
}
\end{lstlisting}

% textidote: ignore begin
\subsubsection{Repository}
% textidote: ignore end

% TODO: Write about the layer

% textidote: ignore begin
\subsubsection{Model}
% textidote: ignore end

% TODO: Write about the layer

% textidote: ignore begin
\subsubsection{Batch processing}
% textidote: ignore end

% TODO: Write about the layer

% textidote: ignore begin
\subsubsection{Service}
% textidote: ignore end

% TODO: Write about the layer
