\section{Testing}\label{sec:testing}
Testing is an important part of production-grade code.
It allows the developers to check that parts of the code behave as expected.
The strictly correct implementation of code testing is~\acrfull{tdd}.
In software development~\acrshort{tdd}, is a method where you design code around tests.
This also means that if one were to truly do~\acrshort{tdd}, you would design tests for all functionality
in your program, and then make code to pass the given tests.
The authors deemed that extensive testing was not strictly needed for this project.
Time constraints were also considered.

Though~\acrshort{tdd} was not used for this project, there was still a need for testing.
It was decided that unit tests were going to be implemented with a focus on the backend software, as
it is here most of the critical functionality is.
When creating unit tests, the authors chose to follow the~\acrfull{aaa} pattern, also called the Build Operate
Check pattern~\cite{clean-code}.


\begin{lstlisting}[
    label={lst:aaa-test},
    caption={Pseudocode that shows the~\acrshort{aaa} pattern for testing.},
    captionpos=b,
]
    test() {
        // Arrange
        SomeState someState = new SomeState(foo);

        // Act
        someState.changeState(bar);

        // Assert
        assert(someState == expectedState);
    }
\end{lstlisting}

As seen in listing~\ref{lst:aaa-test}, to use the~\acrshort{aaa} pattern,
one would first set some state to a known state.
Then you act on the state and transform it through some function, also called unit.
We can finally assert that we get the expected state from the transformation, assert can be seen
as the test passing if the expression returns true, and failing if false.
