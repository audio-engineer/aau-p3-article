\subsection{Cultivating New Ideas}\label{subsec:cultivating-new-ideas}

During the initial consultations with the client NOVA Kaffe~\&~Vinbar, the group gained insight into the problem as seen
from the client's perspective.
For the next stage of the process, the group tried to gain a deeper understanding of the situation by
understanding existing solutions and brainstorming new ideas.

Whenever a new system is to be designed, it is crucial both to be able to objectively listen to the client and
understand the issue from their perspective.
Likewise, it is also crucial to understand whether any old, existing ideas have already solved the problem, since such
ideas could potentially be reused in the given situation.
Furthermore, new ideas need to be cultivated and evaluated in comparison to any existing solutions or ideas.
During this stage of the process, it is of vital importance to continuously brainstorm and evaluate ideas in
collaboration with the client, since, without such a collaboration, any prospective ideas might be too general or
abstract~\cite[32]{mathiassen2018}.

Therefore, the group conducted additional meetings with the client, where possible solutions were discussed.
One of the questions that the group asked the owners of NOVA was whether they were familiar with the Google Analytics
suite or Microsoft Power BI\@.
Both those pieces of software are extremely well-established in the industry and widely used, and can be categorized as
``old ideas'' that serve as \textit{exemplars}~\cite[33]{mathiassen2018} and form a potential solution for the problem.
Interestingly, the owners answered that they were familiar with the named software, but thought of it as either being
too complex to use, as in the case of Microsoft Power BI, or too general as Google Analytics.
Additionally, they expressed that neither of them had the necessary knowledge of data analysis or time to take care of
the prerequisite setup to be able to leverage the power of professional analysis software.
Rather, they hoped that a custom piece of software could be made with simplicity, accessibility and user experience in
mind.

Now that the usage of existing solutions was eliminated, at least directly using them, one of the next questions was
whether the client already had some preconception of what such a custom piece of software could look like.
The client answered that they use an Apple iPad for all the daily operations of the café such as ordering inventory,
receiving payments through the~\acrfull{epos} system, managing staff, etc., and that they would love to be able to use
the custom software on the iPad.
Another requirement was that the software would present the business data through easy-to-understand data visualizations
that don't require knowledge of data analysis concepts.
The latter requirement was of special importance, since it also became clear that the software would have to be used by
all café staff, including young baristas, so accessibility once again appeared as a determining factor in the
discussions.

Continuing on the topic of preconceived ideas, the owners of NOVA told the group that they were thinking that an iPad
application would be the perfect software solution.
The reasoning behind this idea goes back to the fact that all the café's operations happen on the iPad anyway, so simply
having an additional app would be the most efficient scenario.
Now, the group pointed out a couple of obstacles with this approach: Namely, creating a one-off custom tablet
application is unsustainable, since it requires permission from the Apple App Store to be able to deploy it on the
café's iPad.
Such a permission would come with a fee and additional legal requirements, which the client didn't consider when first
discussing this idea internally before the meeting with the group.
So the group proposed an alternative: The application could easily run in the iPad's web browser, as a web application.
The owners of NOVA saw this proposal as a good balance between using the existing iPad infrastructure and ordering a
software solution that is economical, realistic and maintainable.

Next, the question of where the data to be analyzed would come from, and how it would be imported into the system
came up.
The primary source of all data, as the client explained, would be the business's~\acrshort{epos} system.
But the client also pointed out that a startup built the~\acrshort{epos}, which, due to its simple and new design,
lacked data exchange methods such as a public API, or similar.
The~\acrshort{epos} features data export in the form of CSV files, and, as the client explained, a requirement for any
future system would be that it can import those files easily and directly from the user interface.
Furthermore, because the data exported through CSV files from the~\acrshort{epos}, e.g., includes product IDs but no
product names, the data should be editable through the system's user interface.
This would facilitate that attributes such as product names can be added and mapped to file-imported product IDs.

And finally, since the two owners of the café were the sole administrators but not only employees of the business, an
additional requirement was placed that system admins should be able to add more users to the system, e.g., baristas.
This would enable the baristas to access the user interface using their own accounts, rather than relying on a primary
account, or creating an unauthenticated system.
Those users should have admin-editable and -removable accounts, so that the owners could administer the system according
to the current employees.

To conclude, the following requirements were deducted from the client meetings:
\begin{itemize}
    \item Easy-to-understand user interface.

    \item Data analyzed through visualizations.

    \item Target device and technology: iPad browser.

    \item Data imported into the system via a file.

    \item Data can be edited.

    \item Only owners (admins) can add, remove or edit other users.
\end{itemize}
