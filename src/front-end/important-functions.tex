% textidote: ignore begin
\subsection{Important Front-End Functions}\label{subsec:important-front-end-functions}
% textidote: ignore end

When discussing the front-end functions of the application, there are several key functions that are important to the
functionality.
The focus in this section is on the functions related to the charting and data visualization aspects of the application.
That includes the Nivo library components and the data fetching functions.

% textidote: ignore begin
\subsubsection{Charting functions}\label{subsubsec:charting-functions}
% textidote: ignore end

The Nivo library provides a number of options that allows the group to customize the charts to fit the application's
design.
Listing~\ref{lst:heatmap-chart} shows one of the components that the group uses to visualize the data, a heatmap.
Any component can adapt to the layout of the application and the data that is being visualized.
The only required prop is the data prop, which is an array of objects that contain the data to be visualized.
Everything else is optional and can be customized to fit the application's design.
The listing shows the customization done to the heatmap component, with custom margins, padding, colors, legend and
tooltip.
The design can be seen on Figure~\ref{fig:design-final-dash1} in Section~\ref{subsec:final-design}.
The rest of the components are similar in customization, however some options may differ depending on the component.

\begin{lstlisting}[
    style=tsxAndTypescript,
    caption=Chart visualization in the Heatmap component.
    Dots (...) represent skipped code.,
    label={lst:heatmap-chart}]
const DailyHourlySales = (): ReactElement => {
  ...
  return (
    <ResponsiveHeatMap
      data={data}
      margin={{ top: 60, right: 90, bottom: 60, left: 90 }}
      forceSquare={false}
      xInnerPadding={0.05}
      yInnerPadding={0.05}
      borderRadius={6}
      axisTop={{
        tickSize: 5,
        tickPadding: 5,
      }}
      axisLeft={{
        tickSize: 5,
        tickPadding: 5,
      }}
      colors={{
        type: "sequential",
        scheme: "blues",
      }}
      emptyColor="#555555"
      inactiveOpacity={1}
      legends={[
        {
          anchor: "bottom",
          translateX: 0,
          translateY: 30,
          length: 400,
          thickness: 8,
          direction: "row",
          tickPosition: "after",
          tickSize: 5,
          tickSpacing: 4,
          tickOverlap: false,
          titleAlign: "start",
          titleOffset: 4,
        },
      ]}
      tooltip={({ cell }: TooltipProps<HeatMapDatum>) => {
        return (
          <div
            style={{
              background: "white",
              padding: "9px 12px",
              border: "1px solid #ccc",
            }}
          >
            <div>
              {cell.formattedValue} sale(s) on {cell.serieId}s at {cell.data.x}
            </div>
          </div>
        );
      }}
    />
  );
};
\end{lstlisting}

% textidote: ignore begin
\subsubsection{Data fetching functions}\label{subsubsec:data-fetching-functions}
% textidote: ignore end

The data prop needed for the charting components is fetched from the back end using a REST query.
Listing~\ref{lst:heatmap-data} shows how the data is fetched for the heatmap component.
The data is fetched using the~\texttt{useQuery} hook from the React Query library.
The query key is an array containing the query name and the formatted date range.
The query function is an asynchronous function that fetches the data from the back end.
That data is then transformed using the select function to fit the specific format that's needed for the heatmap
component.

\begin{lstlisting}[
    style=tsxAndTypescript,
    caption=Data fetching in the Heatmap component.
    Dots (...) represent skipped code.,
    label={lst:heatmap-data}]
const DailyHourlySales = (): ReactElement => {
  const { formattedDateRange } = useDateRange();

  const { data } = useQuery({
    queryKey: ["dailyHourlySales", formattedDateRange],
    queryFn: async () => getDailyHourlySales(formattedDateRange),
    enabled: null !== formattedDateRange.from || null !== formattedDateRange.to,
    select: transformToHeatmapData,
  });
  ...
\end{lstlisting}

Listing~\ref{lst:data-fetching} shows the function that fetches the data from the back end.
It creates a query string with the start and end date and sends a GET request to the back end.
The response is returned as a promise.

\begin{lstlisting}[
    style=tsxAndTypescript,
    caption=GetDailyHourlySales function.,
    label={lst:data-fetching}]
const getDailyHourlySales = async (
  formattedDateRange: FormattedDateRange,
): Promise<PaginatedResponse<EmbeddedDailyHourlySales>> => {
  const { from, to } = formattedDateRange;

  const response = await client.get<
    PaginatedResponse<EmbeddedDailyHourlySales>
  >(`/api/daily-hourly-sales?startDate=${from}&endDate=${to}`);

  return response.data;
};
\end{lstlisting}

Listing~\ref{lst:data-transofrm} shows the function that transforms the data from the back end to the format required
for the heatmap component.
The function takes the response from the back end and transforms it into an array of objects.
Each object contains an ID and an array of data points.
The data points are objects that contain the x and y values for the heatmap.
They are sorted by the x value, and the objects are sorted by the ID value.
The function then returns the array of objects.

\begin{lstlisting}[
    style=tsxAndTypescript,
    caption=TransformToHeatmapData function.,
    label={lst:data-transofrm}]
const transformToHeatmapData = (
  response: PaginatedResponse<EmbeddedDailyHourlySales>,
): HeatMapSerie<HeatMapDatum, object>[] => {
  const array = response._embedded.dailyHourlySalesDtoes.map((day) => {
    const data = day.hourlySales.map((hourly) => ({
      x: getLocalizedTime(hourly.hour),
      y: hourly.totalSales,
    }));

    data.sort(
      (
        firstElement: Readonly<HeatMapDatum>,
        secondElement: Readonly<HeatMapDatum>,
      ) => {
        return (
          dayjs(firstElement.x, "HH:mm").unix() -
          dayjs(secondElement.x, "HH:mm").unix()
        );
      },
    );

    return {
      id: getLocalizedDate(day.date),
      data,
    };
  });

  array.sort(
    (
      firstElement: HeatMapSerie<HeatMapDatum, object>,
      secondElement: HeatMapSerie<HeatMapDatum, object>,
    ) => {
      return (
        dayjs(firstElement.id, "DD.MM.YYYY").unix() -
        dayjs(secondElement.id, "DD.MM.YYYY").unix()
      );
    },
  );

  return array;
};
\end{lstlisting}
