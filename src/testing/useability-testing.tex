\subsection{Usability Testing}\label{subsec:usability-tests}

User tests have been conducted, as they allow for testing of how convenient
and useful a function is presented to the user.
For this project, the authors decided to conduct two rounds of user tests.
The first round was planned to spot any major oversights in the design.
If any issues were caught at this point in time, the authors would have approximately one and a half month
to fix the issues, this is also why there was more focus on general functionality instead of details.
Here the authors had already discussed and agreed with Nova on the hard requirements for the project.
It was not possible to conduct this test with people from Nova as the schedules did not line up with the authors.
However, because the hard requirements were decided that it would also be useful to get feedback
from an outside perspective.
The second round of user testing was done with the help of a barista from the Nova Café.
Here the focus was more on details and how the flow of the page was experienced, this is mostly because there was
a time limit for how much the authors could change.

\subsubsection{User test 1}\label{subsubsec:user-test-1}

The first round took place on~\DTMdisplaydate{2024}{11}{1}{-1}.
This test was conducted as a workshop at Aalborg university where other students also from the software major,
both first Semester students and third Semester.
At the workshop, the authors presented Nova and the problem the company is facing, which is described in
Section~\ref{subsec:the-current-situation}.

When a user accesses the Nova dashboard, they were presented with a main page where it was only possible to see
one diagram at a time.
The authors' initial idea was that, as it will be used on an iPad, it made sense to not clutter or compress too
much information at the same time.
The way you would switch between diagrams was to click on arrows besides the diagram and then switch until you had
the correct diagram.
This idea was challenged by multiple people at the workshop, who mentioned that as this was a dashboard, we should
have more information on the main page.
It was also discussed that the arrows to switch between diagrams were not needed and added extra steps to achieve
the main goals of the dashboard.

Another important part of the design that the authors got a lot of feedback on was the way the user would choose a
timeframe for the data shown.
The first iteration of the design involved a grid of days that would show the selected year,
see Figure~\ref{fig:lofi-prototype}.
Although the idea was viable, it became more questionable when taken into account that it was to be used on an iPad.
The main problem with the approach was that it could feel weird or annoying to use when wanting to have specific dates
for the data.
To fix the issues with the grid design, a different solution was agreed on, see Section~\ref{subsec:final-design}.

Overall, the feedback received through the workshop allowed the authors to make major changes to design or
functionality, while still having enough time to finish the product.

\subsubsection{User test 2}\label{subsubsec:user-test-2}
The second interview was designed as a think aloud exercise.
The chosen barista did not want to have their name published, and so they will simply be given an alias called Mike.
The structure of the interview was set up so that Mike and two of the authors were sitting around a table,
where the authors acted as interviewers.
The first interviewer sat by the side of Mike and helped if he had any questions.
The other interviewer was to write down any noteworthy points Mike made while thinking aloud,
or to act as an observer.
After Mike had looked at all the different aspects of the website, the interviewers then went back to points of interest
to get clarifications from Mike on why he mentioned certain aspects.
The interview took place at the Nova Café on~\DTMdisplaydate{2024}{11}{27}{-1}, and took approximately 30 minutes.

The following text will summarize most of the parts Mike found issue with, or aspects he did not understand immediately.
The first thing Mike mentioned was that the website was in English, although he did not have a problem understanding it.
In a follow-up question in regard to language support, he clarified that he did not think it was a necessary
functionality for the product, but that it would be nice.

After some comments on the visuals of the design which Mike thought looked good, he navigated to the sales page.
Here he was slightly overwhelmed with the data shown, and the interviewer had to spend some time explaining the
individual entries.

Mike reached the settings page last, here he took some time to create some categories, but he did not see why this was
a feature initially.
The interviewer explained that this information was needed for one of the diagrams and when showed how his changes
affected the diagram, Mike then said he understood what it did and why it was useful.
Mike did not connect the diagram with the settings page intuitively,
as they did not have a strong connection visually to each other.

A core feature that Mike found convoluted was the file upload as he had not worked a lot with this aspect of
their~\acrshort{epos}.
He first uploaded the wrong file, and the interviewer had to explain the difference before he
did it correctly.
Mike said that the file upload was the most challenging part of the website, and that it was not very user-friendly.

Overall, Mike thought that the project had fulfilled Nova Café needs and that it mostly worked well, the biggest problem
pointed out by Mike was the file upload.
However, as explained in Section~\ref{subsec:cultivating-new-ideas}, there was not a choice in this aspect.

