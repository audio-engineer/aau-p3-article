% textidote: ignore begin
\subsection{Front-End Libraries}\label{subsec:front-end-libraries}
% textidote: ignore end

The group uses several front-end libraries to aid in the development of the UX and UI of the application.
The use of these libraries keeps the scope of the project focused on the core functionality of the application, allowing
the group to deliver a product that efficiently meets the requirements of the project.

% textidote: ignore begin
\subsubsection{Material-UI}\label{subsubsec:material-ui}
% textidote: ignore end

To make the user interface appealing, the group uses Material-UI, a React component library that implements Google's
Material Design~\cite{material-ui}.
It provides a set of components that follow the Material Design guidelines, which makes it easy to create a consistent
and visually appealing user interface.
This library is used to create the core components of the application, such as buttons, text fields, and data grids.

% textidote: ignore begin
\subsubsection{Nivo}\label{subsubsec:nivo}
% textidote: ignore end

Nivo is a React component library that provides a set of customizable chart components~\cite{nivo2024}.
From the design phase, the group agreed to use specific types of charts to visualize the data in the application.
There are a lot of React libraries that provide chart components, but many of them did not meet the group's requirements.
They were either missing important charts or had subpar UX or UI.\@
However, Nivo met the group's requirements as it provided a set of customizable chart components that were
responsive, interactive and visually appealing.
The charts were easy to implement and fit in with the Material Design that the application was following.
