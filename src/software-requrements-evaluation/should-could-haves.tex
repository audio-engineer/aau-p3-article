%textidote: ignore begin
\subsection{Should-Haves and Could-Haves}\label{subsec:should-haves-and-could-haves}
% textidote: ignore end

In this section, we will evaluate the should-haves and could-haves requirements.
A lot of the should-haves have been implemented, and the system is functional with them.
This section will only cover the requirements that have not been implemented.

\begin{itemize}
    \item \textbf{Delete an order:}
    Deleting an order was a should-have requirement, but it has not been implemented.
    Once again, this requirement was not prioritized as highly as some other should-haves.
    The reason for this is that it is not at all that important to the company to be able to delete an order.
    As the company has an~\acrshort{epos} system that can already delete orders, so for that reason, it was not a
    priority for the system to be able to delete orders.

    \item \textbf{Update an order:}
    The system could be improved by having the ability to update an order.
    Currently, the system only allows the user to view the orders and then look at the details of the order.
    The user cannot update the order, which could be a useful feature if something goes wrong with the order or is not
    displayed correctly.
    This feature was not implemented as it was not a priority for the systems' functionality, and it was labeled as a
    could-have in Section~\ref{subsec:software-requirements-specification}.
\end{itemize}
